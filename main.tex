\documentclass[eng,printmode]{mgr}
\usepackage{polski}
\usepackage{url}
\usepackage{listings}
\usepackage[utf8]{inputenc}
\usepackage[T1]{fontenc}
\usepackage{scrextend}
\usepackage{graphicx}
\usepackage{subfigure}
\usepackage{psfrag}
\usepackage{float}
\usepackage{amsmath}
\usepackage{amsfonts}
\usepackage{enumitem}
\usepackage{supertabular}
\usepackage{array}
\usepackage{tabularx}
\usepackage{hhline}
\usepackage{graphicx}
\usepackage{showlabels}
\graphicspath{ {images/} }
\newcommand{\R}{I\!\!R}
\newtheorem{theorem}{Twierdzenie}[section]

\title{Porównanie metod wykrywania obserwacji odstających w danych teleinformatycznych}
\engtitle{Comparsion of outliers detecting methods in ICT data}
\author{Maciej Bakowicz}
\supervisor{prof. dr hab. inż. Krzysztof Walkowiak, W-4}

\field{Teleinformatyka (TIN)}
\specialisation{Projektowanie Sieci Teleinformatycznych (TIP)}


\begin{document}
\bibliographystyle{plain}
\maketitle 

\tableofcontents 

\chapter{Wstęp}  
\section{Cel pracy}
Celem niniejszej pracy jest zaprojektowanie oraz zaimplementowanie systemu, który w efektywny sposób wykryje błędy (tu: znaczące odstępstwa od normy w przedziale lokalnym) w danych tyczących się infrastruktury teleinformatycznej (głównie stacje nadawczo-odbiorcze BTS - Base Transceiver Station). W czasie realizacji pracy zostaną zaimplementowane istniejące oraz własne algorytmy a następnie porównane między sobą w celu wybrania sposobu (algorytmu) optymalnego. Zaprojektowany system nie powinien sam ingerować w dane (usuwać, zmieniać), gdyż dane nie są obarczone informacjami dodatkowymi (jak chociażby przerwy w działaniu stacji bazowych) przez co nie ma wystarczającej wiedzy potrzebnej do stwierdzenia przyczyny wystąpienia błędu. W końcowej fazie projektu system powinien móc wskazać błędne dane a następnie przedstawić użytkownikowi możliwości, gdzie w zależności od natury, charakteru użytkownik mógłby je usunąć lub spróbować naprawić (ręcznie lub za pomocą algorytmów uczenia maszynowego). 

\section{Zakres pracy}
\subsection{Przygotowanie danych}
Pierwszym etapem niniejszej pracy jest zebranie rzeczywistych danych od firmy zajmującej się ich zbieraniem i przetwarzaniem.  \\
Wartości oraz opisy pewnych cech tych danych są w dużej mierze tajne i zawierają poufne informacje, dlatego przedstawienie ich w niezmienionej formie byłoby poważnym naruszeniem umowy oraz klauzuli poufności zawartych z firmą. \\
Dlatego kolejnym krokiem jest zaprogramowanie algorytmu, programu, który będzie w stanie zmienić strukturę tych danych, w szczególności przypisać nazwy rzeczywistych operatorów oraz identyfikatorów urządzeń zbierających dane do nazw losowych, którymi posługiwać będzie można się w sposób swobodny w dalszych etapach pracy. \\
Zdobyte dane zawierają również szereg dodatkowych informacji, które powinny zostać odrzucone ze względu na małą wartość informacyjną i zachowanie ich jest niepotrzebne do realizacji celu pracy. Dlatego kolejnym i ostatnim krokiem w przygotowaniu danych jest stworzenie nowych struktur, tabel i schematów a następnie przekopiowanie tych wartości, które są niezbędne. Ilość danych (około pół miliona wierszy) jest na tyle duża, że wykorzystanie mechanizmów do zarządzania Big Data jest niezbędne \cite{cassandra}\cite {cassandra-driver}.

\subsection{Wybór technologii i algorytmów}
Po utworzeniu struktur i wypełnieniu ich danymi kolejnym krokiem jest wybranie istniejących już algorytmów wykrywania obserwacji odstających \cite{outliers-basic} a następnie ich implementacja w wybranym języku programistycznym lub wykorzystanie rozwiązań już wcześniej zaimplementowanych. Ilość zaimplementowanych algorytmów zależy w dużej mierze od stopnia trudności w ich implementacji oraz testowaniu a także od ilości dostępnych materiałów na ich temat. W obrębie wyboru algorytmów znajdzie się zatem nie tylko ich wyszukanie i implementacja ale także wstępna segregacja wraz z odrzuceniem tych mniej obiecujących czy nie pasujących do koncepcji. Część zaimplementowanych algorytmów może w końcowym etapie nie być brane pod uwagę ze względu właśnie na ich mniejszą użyteczność. \\
Najbardziej obiecującą technologią jest język Python \cite{python} ze względu na dużą ilość bibliotek do tworzenia struktur i schematów danych \cite{pandas} a także działań matematycznych co uproszcza proces pisania algorytmów \cite{numpy}.

\subsection{Napisanie własnego algorytmu}
Po zaimplementowaniu i przetestowaniu istniejących już rozwiązań oraz analizie ich wyników zostanie zaimplementowany własny algorytm, który wykorzystuje autorskie rozwiązania. Aby w procesie porównywania algorytmów wyniki były rzetelne i porównywalne rozwiązania te muszą wykorzystywać podobne mechaniki co wcześniej wybrane algorytmy \cite{isolation-forest}\cite{novelty}. 

\subsection{Stworzenie raportu porównawczego}
Końcowym etapem pracy jest napisanie raportu porównawczego oraz wybór cech, które będą ze sobą porównywane. Raport ten zawiera szereg porównań między kolejnymi algorytmami wraz ze wskazaniem tego optymalnego w zależności od przypadku użycia. Algorytmy są testowane kilkukrotnie na różnych zestawach danych (różniącymi się charakterystykami oraz wartościami). \\
Na potrzeby lepszej wizualizacji porównań oraz działania poszczególnych algorytmów do raportu zostaną dołączone spore ilość wykresów, data gramów oraz tabel \cite{react}\cite{react-chart}.


\bibliography{bibliografia}
\end{document}

