\documentclass[eng,printmode]{mgr}
\usepackage{polski}
\usepackage{url}
\usepackage{listings}
\usepackage[utf8]{inputenc}
\usepackage[T1]{fontenc}
\usepackage{scrextend}
\usepackage{graphicx}
\usepackage{subfigure}
\usepackage{psfrag}
\usepackage{float}
\usepackage{amsmath}
\usepackage{amsfonts}
\usepackage{enumitem}
\usepackage{supertabular}
\usepackage{array}
\usepackage{tabularx}
\usepackage{hhline}
\usepackage{graphicx}
\usepackage{showlabels}
\graphicspath{ {images/} }
\newcommand{\R}{I\!\!R}
\newtheorem{theorem}{Twierdzenie}[section]

\title{Projekt i implementacja systemu detekcji błędów w danych teleinformatycznych}
\engtitle{Design and implementation of defect detection system in ICT data}
\author{Maciej Bakowicz}
\supervisor{prof. dr hab. inż. Krzysztof Walkowiak, W-4}

\field{Teleinformatyka (TIN)}
\specialisation{Projektowanie Sieci Teleinformatycznych (TIP)}


\begin{document}

\maketitle 

\tableofcontents 

\chapter{Wstęp}  
\section{Cel pracy}
Celem niniejszej pracy jest zaprojektowanie oraz zaimplementowanie systemu, który w efektywny sposób wykryje błędy (tu: znaczące odstępstwa od normy w przedziale lokalnym) w danych tyczących się infrastruktury teleinformatycznej (głównie stacje nadawczo-odbiorcze BTS - Base Transceiver Station). W czasie realizacji pracy zostaną zaimplementowane istniejące oraz własne algorytmy a następnie porównane między sobą w celu wybrania sposobu (algorytmu) optymalnego. Zaprojektowany system nie powinien sam ingerować w dane (usuwać, zmieniać), gdyż dane nie są obarczone informacjami dodatkowymi (jak chociażby przerwy w działaniu stacji bazowych) przez co nie ma wystarczającej wiedzy potrzebnej do stwierdzenia przyczyny wystąpienia błędu. W końcowej fazie projektu system powinien móc wskazać błędne dane a następnie przedstawić użytkownikowi możliwości, gdzie w zależności od natury, charakteru użytkownik mógłby je usunąć lub spróbować naprawić (ręcznie lub za pomocą algorytmów uczenia maszynowego). 

\end{document}
